\documentclass[12pt]{beamer}
\usetheme{Madrid}
\usefonttheme{serif}

% ===== XeLaTeX 專用 =====
\usepackage{iftex}
\ifPDFTeX
  \errmessage{請用 XeLaTeX 編譯(latexmk -xelatex)}
\fi
\usepackage[ruled,vlined,linesnumbered]{algorithm2e}
%               ^^^^^^^^^^^^^^^  要有這個% 行號樣式調整(讓它更像 paper)
\SetAlgoNlRelativeSize{-1} % 行號縮小一點
\SetNlSty{}{}{\kern0.4em}  % 只顯示數字,不加括號,後面留一點空隙
% ===== 字體:英=Times New Roman;中=Noto Serif CJK TC =====
\usepackage{fontspec}
\usepackage{xeCJK}
%\setmainfont{TeX Gyre Pagella}
\setmainfont{Times New Roman}
\setCJKmainfont{Noto Serif CJK TC}
\xeCJKsetup{PunctStyle=plain}

% ===== 常用套件 =====
\usepackage{amsmath,amssymb,mathtools}
\usepackage{graphicx,booktabs,hyperref,multicol}

\hypersetup{colorlinks=true,linkcolor=.,urlcolor=.,citecolor=.}

% ===== Beamer 外觀 =====
% ===== Beamer 外觀 =====
% ===== Beamer 外觀 =====
\setbeamertemplate{itemize items}[ball]
\setbeamertemplate{enumerate items}[circle]
\setbeamertemplate{caption}[numbered]
\setbeamertemplate{navigation symbols}{}

% --- footer 顏色設定 ---
\setbeamercolor{footleft}{fg=white,bg=blue!60!black}   % 深藍底 + 白字
\setbeamercolor{footright}{fg=blue!70!black,bg=blue!10!white} % 很淺藍底 + 深藍字


% --- footer 版面:左邊 Section,右邊頁碼,各佔一半 ---
\setbeamertemplate{footline}{%
  \leavevmode%
  \hbox{%
    % 左:目前 section 名稱
    \begin{beamercolorbox}[wd=.5\paperwidth,ht=2.5ex,dp=1.2ex,center]{footleft}%
      \insertsection
    \end{beamercolorbox}%
    % 右:頁碼/總頁數
    \begin{beamercolorbox}[wd=.5\paperwidth,ht=2.5ex,dp=1.2ex,center]{footright}%
      \insertframenumber{} / \inserttotalframenumber
    \end{beamercolorbox}%
  }%
  \vskip0pt%
}
% ===== Section 開頭自動顯示這張 =====
\AtBeginSection[]{
  \begin{frame}{Outline}
    \tableofcontents[currentsection]
  \end{frame}
}
% ===== 文件資訊 =====
\title{Accelerating \textit{ab initio} real-space electronic structure calculations\\
for low-dimensional materials using an atom-sphere grid truncation method}
\author[ChihenTai]{Speaker \quad 戴致恩}
\date{\today}

\begin{document}

% ===== 封面 =====
\begin{frame}[plain]
  \centering
  \vspace{3mm}

  % paper 標題截圖
  \includegraphics[width=0.95\textwidth]{information.png}

  \vspace{6mm}
  \includegraphics[width=0.80\textwidth]{instutition.png}

  % 期刊圖片(縮小放在下面)
  \includegraphics[width=0.30\textwidth]{journals.png}

  \vspace{6mm}

  % 講者資訊(名字比較小)
  \small
  Speaker: 戴致恩 \\[2mm]
  2025/11/20

\end{frame}

% ===== 大綱 =====
% \begin{frame}{Outline}
%   \tableofcontents
% \end{frame}

% ==============================
%        Introduction
% ==============================
\section{Introduction}
\begin{frame}{Hohenberg–Kohn Theorem}

\vspace{0.5em}

If $E$ is the lowest energy of the system, then $E$ is a functional of the electron density:
\[
E = F[n]
\]

\vspace{1.0em}

\begin{itemize}
  \item We avoid solving the full many-electron wavefunction 
      $\Psi(\mathbf r_1,\ldots,\mathbf r_N)$ and work with the electron density function $n(\mathbf r)$
\end{itemize}

\vspace{1.0em}

DFT (Hohenberg–Kohn):
\[
E = F[n(\mathbf r)]
\]

\end{frame}

\begin{frame}{Kohn–Sham Equation}

\vspace{0.2em}

Kohn–Sham equation is a single-particle equation\\
electron–electron interactions are included through the density


\vspace{0.2em}

\[
\hat{H}_{\mathrm{KS}}[\rho]\;|\psi_n\rangle = \varepsilon_n\,|\psi_n\rangle
\]

\vspace{0.2em}

\begin{small}
where $|\psi_n\rangle$ and $\varepsilon_n$ are the $n$th KS eigenstate and eigenvalue
, $\rho$ is the electron density, 
$\hat{H}_{\mathrm{KS}}$ is the Hamiltonian operator\\
\end{small}
\vspace{0.4em}

Electron density:
\[
  \rho(\mathbf{r}) = \sum_{n} f_n\,|\psi_n(\mathbf{r})|^2
\]
\vspace{0.4em}

\small
where $f_n$ is the Fermi–Dirac occupation function\\
$\psi_n(\mathbf{r}) = \langle \mathbf{r} | \psi_n \rangle$ is the real-space wavefunction
\normalsize
\end{frame}
\begin{frame}{Kohn–Sham Equation}
Hamiltonian in the Kohn–Sham equation:
\vspace{-1em}
\begin{center}
\[
  \hat{H}_{\mathrm{KS}}[\rho]
  =
  \hat{T}
  + \hat{V}_{\mathrm{ion}}
  + \hat{V}_{\mathrm{H}}[\rho]
  + \hat{V}_{\mathrm{XC}}[\rho]
\]
\end{center}
\begin{align*}
 &\hat{T}: \text{ kinetic energy operator} \\
 &\hat{V}_{\mathrm{ion}}: \text{ ionic potential} \\
 &\hat{V}_{\mathrm{H}}[\rho]: \text{ Hartree potential} \\
 &\hat{V}_{\mathrm{XC}}[\rho]: \text{ exchange–correlation potential}
\end{align*}

Self-consistent solution procedure:
Iterate KS equations until both the electron density and the  
total energy are converged

\end{frame}
\begin{frame}{Real–space KS-DFT}

\begin{itemize}
    \item KS wavefunctions, electron density, and potentials
          are represented directly on a spatial grid
          \vspace{0.5em}

    \item Accuracy is controlled by the real-space resolution (grid spacing)
          \vspace{0.5em}

    \item Naturally highly parallelizable across many CPU cores
\end{itemize}

\end{frame}
\begin{frame}{Uniform Real–space Grid Representation}

Wavefunction on a uniform grid:
\[
  |\psi_n\rangle \;\approx\;
  \sum_{\mu=1}^{N_b} \omega_\mu\,\psi_{n\mu}\,|\mathbf r_\mu\rangle
\]

\small
where\quad
$\psi_{n\mu} \equiv \psi_n(\mathbf r_\mu)
= \langle \mathbf r_\mu | \psi_n \rangle$
\normalsize

\vspace{0.5em}

\begin{center}
\small
$\{\mathbf r_\mu\}$: grid points \qquad
$\omega_\mu$: volume weight of each point
\end{center}
\vspace{0.5em}
Electron density on the grid:
\[
  \rho(\mathbf r_\mu) = \sum_{n} f_n\, |\psi_{n\mu}|^2
\]

\end{frame}
\begin{frame}{Discrete Kohn–Sham Equation}

Discrete Kohn–Sham equation:
\[
  \mathbf{H}[\rho]\,\Psi_n = \epsilon_n\,\Psi_n
\]

\small
\noindent
where\quad
$\Psi_n \equiv [\,\psi_{n1},\,\psi_{n2},\,\ldots,\,\psi_{nN_b}\,]^{T}$
\;\;is the $n$th discrete wavefunction (eigenvector).
\normalsize

\vspace{0.5em}

Hamiltonian matrix elements:
\[
\begin{aligned}
H_{\mu\nu}[\rho]
&\equiv H[\rho](\mathbf r_\mu, \mathbf r_\nu)
= \langle \mathbf r_\mu \,|\, \hat H_{\mathrm{KS}}[\rho] \,|\, \mathbf r_\nu \rangle\,\omega_\nu \\[0.5em]
&= T(\mathbf r_\mu, \mathbf r_\nu)
+ V_H[\rho](\mathbf r_\mu)\delta_{\mu\nu}
+ V_{XC}[\rho](\mathbf r_\mu)\delta_{\mu\nu} \\[0.5em]
&\quad +
\underbrace{
V_{\mathrm{loc}}(\mathbf r_\mu)\delta_{\mu\nu}
+ V_{\mathrm{nl}}(\mathbf r_\mu,\mathbf r_\nu)
}_{\text{$V_{\mathrm{ion}}(\mathbf r_\mu,\mathbf r_\nu)$}}
\end{aligned}
\]

\end{frame}
\begin{frame}{Uniform Grids in Real-space DFT}

\begin{itemize}
    \item A uniform 3D grid is constructed in the simulation cell
          \vspace{0.5em}

    \item Number of grid points is determined by lattice vectors
          $(\mathbf{a}, \mathbf{b}, \mathbf{c})$ and the spacing $h$
          \vspace{0.5em}

    \item Grid points fill the entire simulation cell,
          including large vacuum regions
          \vspace{0.9em}

    \item Works for both boundary types:
          \begin{itemize}
              \item PBC: periodic systems (solids, 2D materials, nanotubes)
              \item DBC: isolated systems (molecules, clusters)
          \end{itemize}
\end{itemize}

\end{frame}
\begin{frame}{Boundary Conditions in Real–space DFT}

% ---------- PBC ----------
\begin{columns}[T,totalwidth=\textwidth]
  % 左:PBC 文字
  \begin{column}{0.65\textwidth}
    Periodic Boundary Condition (PBC)
    \begin{itemize}
        \item System repeats periodically (like an infinite crystal)
        \item $\psi(\mathbf r + \mathbf R) = \psi(\mathbf r)$
        \item Used for solids, 2D materials, nanotubes
    \end{itemize}
  \end{column}
  % 右:PBC 圖(往上移一點)
  \begin{column}{0.35\textwidth}
    \vspace{-0.6em} % 往上拉
    \begin{flushright}
      \includegraphics[width=0.75\columnwidth]{pbc.png} % 換成你的 PBC 圖檔
    \end{flushright}
  \end{column}
\end{columns}

\vspace{0.6em}

% ---------- DBC ----------
\begin{columns}[T,totalwidth=\textwidth]
  % 左:DBC 文字
  \begin{column}{0.65\textwidth}
    Dirichlet Boundary Condition (DBC)
    \begin{itemize}
        \item Wavefunction is set to zero at the boundary
        \item Used for isolated systems: molecules, clusters
    \end{itemize}
  \end{column}

  % 右:DBC 圖(也往上移一點)
  \begin{column}{0.35\textwidth}
    \vspace{-0.3em} % 往上拉
    \begin{flushright}
      \includegraphics[width=0.75\columnwidth]{dbc.png} % 換成你的 DBC 圖檔
    \end{flushright}
  \end{column}
\end{columns}

\end{frame}

% \begin{frame}{Full Uniform Grids in Real-space DFT}
% Performs poorly in the following cases:
%   \begin{itemize}
%               \item low-dimensional materials (2D sheets, nanotubes)
%               \item isolated clusters (C$_{60}$, nanoparticles)
%               \item porous materials (MOFs, DSSCs)
%           \end{itemize}
% \begin{center}
% \includegraphics[width=0.7\textwidth]{full_grid.png}
% \end{center}
% \end{frame}
%

\begin{frame}{Uniform Grids Are Wasteful}

\begin{columns}[c] % c = 垂直置中,t = 上對齊
  \column{0.67\textwidth}
    Many materials contain large vacuum regions:
    \begin{itemize}
        \item 0D clusters (C$_{60}$, nanoparticles)
        \item 1D nanotubes (SWNT)
        \item 2D layers with vacuum (graphene, tBLG)
        \item 3D porous materials (MOFs, DSSCs)
    \end{itemize}

  \column{0.33\textwidth}
    \centering
    \includegraphics[width=0.7\textwidth]{full_grid_t.jpg}
    % \vspace{0.3em}
    % \scriptsize Full uniform grid covering both atoms and vacuum.
\end{columns}
\vspace{0.6em}
    Problem (atoms are not average distribution):
    \vspace{0.2em}

    \begin{itemize}
        \item Uniform grids waste many points in empty space
        \item Leads to slower computation and high memory usage
    \end{itemize}
\end{frame}

% \begin{frame}{Comparison to Other Basis Sets}

% Atomic orbital basis:
% \begin{itemize}\itemsep4pt
%     \item Small and efficient, but accuracy is limited by basis quality
% \end{itemize}

% \vspace{0.4em}

% Plane-wave basis:
% \begin{itemize}\itemsep4pt
%     \item Accuracy increases with more plane waves. ideal for periodic systems, but large vacuum requires many extra waves
% \end{itemize}

% \vspace{0.4em}

% Real-space finite differences:
% \begin{itemize}\itemsep4pt
%     \item Simple to implement and supports various boundary conditions, but makes useless grids in vacuum regions
% \end{itemize}

% \end{frame}
% \begin{frame}{Existing Methods}

% Finite-element / adaptive grids
% \begin{itemize}
%     \item Dense grids near atoms, sparse in vacuum
%     \item Setup and implementation are complicated
% \end{itemize}

% \vspace{0.8em}

% Wavelet multi-resolution
% \begin{itemize}
%     \item High resolution near atoms, low resolution in empty regions
%     \item Basis construction is complex
% \end{itemize}

% \vspace{0.8em}

% Atomic-orbital basis
% \begin{itemize}
%     \item Naturally localized near atoms
%     \item Accuracy depends strongly on basis choice
% \end{itemize}

% \end{frame}

% ==============================
%        Methodology
% ==============================
\section{Methodology}

% ---------- AST----------
\begin{frame}{Atom-Sphere Truncation (AST)}
        Core concept: Electrons are mainly located around atoms

        \vspace{0.6em}

        \begin{itemize}
            \item Define a cutoff radius $R_c$ around each atom
            \item Works for both PBC and DBC
            \item Fewer grid points and faster SCF steps
            \item Energy difference converges to within 2 meV/atom
        \end{itemize}
        \vspace{0.5em}
          \begin{center}
              \includegraphics[width=0.7\textwidth]{purpose.jpg}
          \end{center}

\end{frame}
\begin{frame}{AST definitions}

AST grid set
\[
\Omega_{\mathrm{AST}}
= 
\bigcup_a 
\left\{
  \mathbf r_\mu : 
  \|\mathbf r_\mu - \mathbf R_a\| \le R_c
\right\}.
\]

\vspace{0.6em}
Truncated wavefunction
\[
\psi_n^\mu =
\begin{cases}
\psi_n(\mathbf r_\mu) & \mathbf r_\mu \in \Omega_{\mathrm{AST}} \\[6pt]
0 & \text{otherwise}
\end{cases}
\]

\vspace{0.6em}

\footnotesize
$\mathbf r_\mu$: grid point \quad
$\mathbf R_a$: atom position \quad
$R_c$: cutoff radius
\normalsize

\end{frame}


\begin{frame}{AST Grids in Real Space}

\begin{itemize}
    \item The AST algorithm keeps only grid points within a cutoff radius $R_c$
          around each atom
    \vspace{0.6em}
    \item This removes most vacuum grid points and reduces the computational cost
\end{itemize}

\begin{center}
\includegraphics[width=0.80\textwidth]{ast_grid.png}
\end{center}

\end{frame}

\begin{frame}{Details of Algorithms and Implementation}

\begin{columns}[T,onlytextwidth]  % T = top align, onlytextwidth 比較好排

  \begin{column}{0.58\textwidth}
    \centering
    \includegraphics[width=0.6\textwidth]{redscf.jpeg}
  \end{column}

  \begin{column}{0.42\textwidth}
    \footnotesize
    \begin{itemize}
        \item Exchange–correlation:  
              PBE functional (via \texttt{libxc})
        \vspace{0.3em}
        \item Eigenproblem solver:  
              CheFSI 
        \vspace{0.3em}
    \end{itemize}
    \vspace{0.5em}
    \centering
    %\includegraphics[width = 0.7\textwidth]{ast_al.png}
\end{column}

\end{columns}

\end{frame}


\begin{frame}{Algorithm 1: Generation of Full Uniform Grids}

\scriptsize
\setlength{\algomargin}{1.8em}

\begin{algorithm}[H]
\DontPrintSemicolon
\KwData{Lattice vectors ($\mathbf{a}, \mathbf{b}, \mathbf{c}$), grid spacing $h$}
\KwResult{$N_b, N_x, N_y, N_z, h_x, h_y, h_z, \{\mathbf r_\mu\}$}

\BlankLine
\LinesNotNumbered
\textit{// Calculate the number of full grids}\;
\LinesNumbered
$N_x = \left\lfloor \dfrac{|\mathbf{a}|}{h} \right\rfloor,\quad
 N_y = \left\lfloor \dfrac{|\mathbf{b}|}{h} \right\rfloor,\quad
 N_z = \left\lfloor \dfrac{|\mathbf{c}|}{h} \right\rfloor$\;
$N_b = N_x N_y N_z$\;

\BlankLine
\LinesNotNumbered
\textit{// Reset the grid spacings}\;
\LinesNumbered
$h_x = |\mathbf{a}|/N_x,\quad
 h_y = |\mathbf{b}|/N_y,\quad
 h_z = |\mathbf{c}|/N_z$\;

\BlankLine
\LinesNotNumbered
\textit{// Loop over 3D grid points}\;
\LinesNumbered
\For{$k = 1$ \KwTo $N_z$}{
  \For{$j = 1$ \KwTo $N_y$}{
    \For{$i = 1$ \KwTo $N_x$}{
      \LinesNotNumbered
      \textit{// Map integer coordinates to in-memory index}\;
      \LinesNumbered
      $\mu = i + N_x (j-1) + N_x N_y (k-1)$\;

      \LinesNotNumbered
      \textit{// Store grid position}\;
      \LinesNumbered
      $\mathbf r_\mu =
        \left(\frac{\mathbf{a}}{N_x},\, \frac{\mathbf{b}}{N_y},\, \frac{\mathbf{c}}{N_z}\right)
        (i,j,k)^T$\;
    }
  }
}
\BlankLine

\textbf{return} $N_b, N_x, N_y, N_z, h_x, h_y, h_z, \{\mathbf r_\mu\}$\;

\end{algorithm}

\end{frame}


\begin{frame}{Algorithm 2: Generation of the atom-sphere truncation
(AST) grid size and coordinates.}
\scriptsize
\setlength{\algomargin}{1.8em}  % 調大一點就會整塊往右移
\begin{algorithm}[H]
\DontPrintSemicolon
\KwData{Full-grid count and positions $N_b,\{\mathbf r_\mu\}$;
       number of atoms $N_a$; atom positions $\mathbf R_a$;
       cutoff radius $R_c$}
\KwResult{$\Omega_{\mathrm{AST}},\, N_b^{\mathrm{AST}}$}

\textit{// Initialize AST grids}\;
$\Omega_{\mathrm{AST}} = \varnothing$\;
$N_b^{\mathrm{AST}} = 0$\;

\For{$\mu = 1$ \KwTo $N_b$}{
  
  \textit{// Select the grid positions}\;
  \For{$a = 1$ \KwTo $N_a$}{
    
    \If{$\|\mathbf r_\mu - \mathbf R_a\| \le R_c$}{
      
      \textit{// Store grid point}\;
      $\Omega_{\mathrm{AST}} = \Omega_{\mathrm{AST}} \cup \{\mathbf r_\mu\}$\;
      $N_b^{\mathrm{AST}} = N_b^{\mathrm{AST}} + 1$\;
    }
  }
}

\Return{$\Omega_{\mathrm{AST}},\, N_b^{\mathrm{AST}}$}

\end{algorithm}

\end{frame}
% ==============================
%        Results
% ==============================
\section{Results}

\begin{frame}{Isolated Cluster Systems}

\begin{itemize}
    \item Tested: C$_{60}$ and three B$_{84}$ structures
    \vspace{0.3em}
    \item Energy difference converges to within 2 meV at $R_c \approx 4.0$ Å
    \vspace{0.3em}
    \item Grid reduction: 35–73\%
    \vspace{0.3em}
    \item Time reduction per SCF step: 41–71\%
\end{itemize}

\begin{center}
\includegraphics[width=0.70\textwidth]{cluster_result.png}
\end{center}

\end{frame}

\begin{frame}{Low-dimensional and Adsorption Systems}

\begin{itemize}
    \item Tested: SWNT (1D), tBLG (2D), B$_{36}$ cluster, hexane/graphene
    \vspace{0.3em}
    \item Energy difference converges to within 2 meV/atom at $R_c \approx 3.6$ Å
    \vspace{0.3em}
    \item Grid reduction: 30–46\%
    \vspace{0.3em}
    \item Time reduction per SCF step: 37–52\%
\end{itemize}

\begin{center}
\includegraphics[width=0.65\textwidth]{extended_results.png}
\end{center}

\end{frame}
\begin{frame}{MOFs}

\begin{itemize}
    \item Systems: IRMOF-1(a) and IRMOF-10(b)
    \vspace{0.3em}
    \item Energy difference converges to within 2 meV/atom at $R_c \approx 3.6$ Å
    \vspace{0.3em}
    \item Grid reduction: 36\% (a)\& 50\% (b)
    \vspace{0.3em}
    \item Time reduction per SCF step: 36\% (a) \& 47\% (b)
\end{itemize}

\begin{center}
\includegraphics[width=0.65\textwidth]{mof_dos.png}
\end{center}
\end{frame}
\begin{frame}{Table : Efficiency comparison of Full grids vs AST}
\scriptsize
\renewcommand{\arraystretch}{1.3}
\setlength{\tabcolsep}{10pt}  % *收一點點,比 4pt 更緊,仍然不擠壓

\begin{table}
\centering
\begin{tabular}{l c cc cc}
\toprule
\textbf{Test system} &
\textbf{Atoms} &
\multicolumn{2}{c}{\textbf{Number of grid points}} &
\multicolumn{2}{c}{\textbf{SCF steps [time(s)/step]}} \\
\cmidrule(lr){3-4} \cmidrule(lr){5-6}
 & & \textbf{Full} & \textbf{AST$^{\mathrm{a}}$} &
     \textbf{Full} & \textbf{AST$^{\mathrm{a}}$} \\
\midrule
% -------- Group 1: isolated clusters --------
C$_{60}$     & 60  & 449k  & 287k  & 30(2.2)   & 30(1.3)   \\
B$_{84}$–1   & 84  & 1802k & 484k  & 32(14.3)  & 31(4.1)   \\
B$_{84}$–2   & 84  & 716k  & 366k  & 32(4.2)   & 28(2.3)   \\
B$_{84}$–3   & 84  & 927k  & 496k  & 32(6.8)   & 31(3.6)   \\
\\[-0.8em]   % 空一組,不要增加太多 vertical space

% -------- Group 2: low-dimensional systems & adsorption --------
SWNT         & 112 & 677k  & 364k  & 41(12.0)  & 44(5.7)   \\
tBLG         & 148 & 592k  & 413k  & 33(10.7)  & 32(6.6)   \\
B$_{36}$     & 196 & 1109k & 602k  & 44(35.7)  & 40(19.1)  \\
Hexane       & 80  & 435k  & 293k  & 36(4.9)   & 34(3.1)   \\
\\[-0.8em]

% -------- Group 3: MOFs --------
IRMOF-1      & 106 & 1161k & 718k  & 85(39.2)  & 70(25.1)  \\
IRMOF-10     & 190 & 2628k & 1319k & 81(80.3)  & 79(42.3)  \\
\bottomrule
\end{tabular}
\vspace{0.4em}
\footnotesize
\\
$^{\mathrm{a}}$Absolute errors for AST are below 2 meV/atom.

\end{table}

\end{frame}

\section{Conclusion}
\begin{frame}{Conclusion}
The AST method improves real-space KS-DFT efficiency:
\begin{itemize}
    \item Reduces 30–73\% of real-space grid points by removing vacuum regions  
\vspace{0.6em}
\item Accelerates SCF calculations by 36–71\%, tested on clusters, nanotubes, 2D materials, and MOFs 

\vspace{0.6em}

\item Maintains energy accuracy within 2 meV/atom using ${R_c}$ $\approx$ 3.6–4.0 Å  

\end{itemize}
\end{frame}


% ==============================
%        結尾
% ==============================
{
\setbeamertemplate{footline}{}  % 只在這一組花括號內關掉 footline

\begin{frame}[plain]
  \vfill
  \centering
  \Large Thank you!\\[12pt]
  \vfill
\end{frame}

}

\end{document}
